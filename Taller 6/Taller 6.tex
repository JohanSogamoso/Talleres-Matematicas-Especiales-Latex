\documentclass[a4paper,11pt,openany]{book}
\usepackage[utf8]{inputenc}
\usepackage[left=2.54cm,top=2.54cm,right=2.54cm,bottom=2.54cm]{geometry}
\usepackage[spanish]{babel}
\usepackage{amsmath}
\usepackage{amsfonts}
\usepackage{amssymb}
\usepackage{graphicx}
\usepackage{color}
\usepackage[usenames,dvipsnames]{xcolor}
\usepackage{pifont}
\usepackage{marvosym}
\newtheorem{teo}{Teorema}
\newtheorem{ejemplo}{Ejemplo}
\newtheorem{defi}{Definición}
\newtheorem{coro}{Corolario}
\newtheorem{prueba}{Prueba}
\newtheorem{exmp}{Example}[section]
\newtheorem{ejer}{Ejercicio}[section]
\def\proof{\paragraph{\textsf{Demostración.} }}
\def\endproof{\hfill $\blacksquare$ \\}
\usepackage{multirow, array} % para las tablas
\usepackage{multirow}
\usepackage{tabularx}
\usepackage{float} % para usar [H]
\usepackage{tikz}
\usepackage[all]{xy}
\usepackage{cancel}
\usetikzlibrary{positioning}
\usepackage{enumitem}
\newcommand*{\itembolasazules}[1]{% bolas 3D
\footnotesize\protect\tikz[baseline=-3pt]%
\protect\node[scale=.7, circle, shade, ball
color=green]{\color{white}\Large\bf#1};}
\usepackage{tcolorbox} 
\tcbuselibrary{listingsutf8}
\newtcolorbox[auto counter,number within=section]{example}[2][]
{colback=green!5!white,colframe=green!75!black,fonttitle=\bfseries, title=Ejercicio~\thetcbcounter: #2,#1}
\usepackage{background}
\backgroundsetup{
placement=center,
angle=0,
scale=1.1,
contents= {{\includegraphics{HojaCuadriculada.png}}}
}
 
 
\begin{document}
\begin{titlepage}
 
\begin{center}
\vspace*{-1in}
\begin{figure}[htb]
\begin{center}
\includegraphics[width=7cm]{ETITC.png}
\end{center}
\end{figure}
 
 
{\sc \huge Escuela Tecnológica Instituto Técnico Central (ETITC)}\\
\vspace*{0.15in}
Facultad de sistemas\\
\vspace*{0.6in}
\begin{Large}
\textbf{Taller 6: Sucesiones y Series Complejas.} \\
\textbf{Matem{\'a}ticas Especiales}\\
\end{Large}
\vspace*{0.3in}
\begin{large}
{\bf Autores} \\
 
\ 
 
Sergio Alejandro Enrrique Caballero Leon\\ 
Johan Alejandro Sogamoso Camacho \\
David Andrés Valero Vanegas \\
\end{large}
\vspace*{0.3in}
 
\end{center}
 
\begin{center}
{\bf Presentado a:} \\
 
\ 
 
Carlos Romero \\
 
\
 
Bogot{\'a}, Diciembre de 2022.
\end{center}
 
\end{titlepage}

\newpage

\definecolor{ao(english)}{rgb}{0.0, 0.5, 0.0}

\definecolor{mediumviolet-red}{rgb}{0.78, 0.08, 0.52}

\definecolor{mint}{rgb}{0.24, 0.71, 0.54}

\begin{center}
\textbf{Sucesiones Complejas}
\end{center}

En los ejercicios \textbf{(\,1\,)} y \textbf{(\,2\,)} escriba los primeros cinco términos de la {\it sucesión} dada.\\

\textcolor{ao(english)}{(\,1\,)} $\bf{\{5\,i^{n}\}}$.

$$\{5\,i^{n}\}\,=\,5\,i,\,-\,5,\,-\,5\,i,\,5,\,5\,i,\,....$$

\textcolor{ao(english)}{(\,2\,)} $\bf{\{2\,+\,(-\,i)^{n}\}}$.

$$ Z_{1}\,=\,2\,-\,i $$

$$ Z_{2}\,=\,2\,-\,1 \iff Z_{1}\,=\,1  $$

$$ Z_{3}\,=\,2\,+\,i $$

$$ Z_{4}\,=\,2\,+\,1 \iff Z_{4}\,= 3$$

$$ Z_{5}\,=\,2\,-\,i $$

En los ejercicios \textbf{(\,3\,)} al \textbf{(\,6\,)} use le \textcolor{mediumviolet-red}{{\it Teorema} 1} para mostrar si la sucesión dada en cada caso es {\it Covergente ó Divergente}. En caso de ser covergente, muestre a cuál número converge.\\

\textcolor{ao(english)}{(\,3\,)} $\bf{\left\{\dfrac{3\,n\,i\,+\,2}{n\,+\,n\,i}\right\}}$.\\

 Divido cada termino por $"n"$ con el mayor exponente:

 $$\dfrac{\dfrac{3\,n\,i}{n}\,+\,\dfrac{2}{n}}{\dfrac{n}{n}\,+\,\dfrac{n\,i}{n}}\,=\,\dfrac{3\,i\,+\,\dfrac{2}{n}}{1\,+\,i}\,\times\,\dfrac{1\,-\,i}{1\,-\,i}\,=\,\dfrac{3\,i\,+\dfrac{2}{n}\,+\,3\,-\dfrac{2\,i}{n}}{1\,-\,i\,+\,i\,-\,i^{2}}$$\\

$$\displaystyle\lim_{n \to \infty} \dfrac{3\,i\,+\dfrac{2}{n}\,+\,3\,-\dfrac{2\,i}{n}}{2} => \dfrac{3\,i\,+\,0\,+\,3\,+\,0}{2} $$

$\therefore$ la sucesión converge a:

$$\mathnormal{L}\,=\,\dfrac{3}{2}\,+\,\dfrac{3i}{2}$$

\textcolor{ao(english)}{(\,4\,)} $\bf{\left\{\dfrac{n\,i\,+\,2}{3\,n\,i\,+\,5}\right\}}$.\\

 Divido cada termino por $"n"$ con el mayor exponente:
 
 $$\dfrac{\dfrac{n i}{n}+\dfrac{2}{n}}{\dfrac{3n i}{n}+\dfrac{5}{n}} = \dfrac{i+\dfrac{2}{n}}{3i+\dfrac{5}{n}} \times \dfrac{3i-\dfrac{5}{n}}{3i-\dfrac{5}{n}} = \dfrac{\left(i+ \dfrac{2}{n} \right) \left(3i- \dfrac{5}{n} \right)}{ \left( \dfrac{5}{n} \right)^{2} + (3i)^{2} } $$
 
 $$\displaystyle\lim_{n \to \infty} \dfrac{-3+\dfrac{i}{n}-\dfrac{10}{n^{2}}}{\dfrac{25}{n^{2}}-9} => \dfrac{-\,3\,+\,0\,+\,0}{0 -9} $$
 
 $$\mathnormal{L}\,=\,\dfrac{1}{3}$$

\textcolor{ao(english)}{(\,5\,)} $\bf{\left\{\dfrac{(n\,i\,+\,2)^{2}}{n^{2}i}\right\}}$.\\

\textcolor{ao(english)}{\ding{46}} como $z_{n}\,=\,\dfrac{(n\,i\,+\,2)^{2}}{n^{2}i}$ obtener la parte $\mathnormal{Re}$ e $\mathnormal{Im}$.

$$\dfrac{(n\,i\,+\,2)^{2}}{n^{2}i}\,\textcolor{red}{\times\,\dfrac{-\,n^{2}i}{-\,n^{2}i}}\,=\,\dfrac{-\,n^{2}i\,(n\,i\,+\,2)^{2}}{n^{4}}\,=\,\dfrac{-\,n^{2}i\,(-\,n^{2}\,+\,4\,n\,i\,+\,4)}{n^{4}}$$

$$=\,\dfrac{n^{4}\,i\,+\,4\,n^{3}\,-\,4\,n^{2}\,i}{n^{4}}\,=\,\underbrace{\dfrac{4\,n^{3}}{n^{4}}}_{\mathnormal{Re}\,(z_{n})}\,+\,i\,\underbrace{\dfrac{n^{4}\,-\,4\,n^{2}}{n^{4}}}_{\mathnormal{Im}\,(z_{n})}$$

$$\displaystyle\lim_{n \to \infty}\,\mathnormal{Re}\,(z_{n})\,=\,\displaystyle\lim_{n \to \infty}\,\dfrac{4\,n^{3}}{n^{4}}\,=\,\displaystyle\lim_{n \to \infty}\,\dfrac{\dfrac{4\,n^{3}}{n^{4}}}{\dfrac{n^{4}}{n^{4}}}\,=\,\displaystyle\lim_{n \to \infty}\,\dfrac{4}{n}\,=\,0$$

$$x\,=\,0$$

$$\displaystyle\lim_{n \to \infty}\,\mathnormal{Im}\,(z_{n})\,=\,\displaystyle\lim_{n \to \infty}\,\dfrac{n^{4}\,-\,4\,n^{2}}{n^{4}}\,=\,\displaystyle\lim_{n \to \infty}\,\dfrac{\dfrac{n^{4}}{n^{4}}\,-\,\dfrac{4\,n^{2}}{n^{4}}}{\dfrac{n^{4}}{n^{4}}}\,=\,\displaystyle\lim_{n \to \infty}\,1\,-\,\dfrac{4}{n^{2}}\,=\,1$$

$$y\,=\,1$$

$\therefore$ la sucesión converge a $\mathnormal{L}\,=\,x\,+\,i\,y$

$$\mathnormal{L}\,=\,i$$

\textcolor{ao(english)}{(\,6\,)} $\bf{\{z_{n}\}\,=\,\left\{\dfrac{4\,n\,+\,3\,n\,i}{2\,n\,+\,i}\right\}}$.\\

Divido cada termino por $"n"$ con el mayor exponente:

$$\dfrac{\dfrac{4n}{n}+\dfrac{3ni}{n}}{\dfrac{2n}{n}+\dfrac{i}{n}} = \dfrac{ 4 +3i }{2+\dfrac{i}{n}} $$

$$\displaystyle\lim_{n \to \infty} \dfrac{ 4 +3i }{2+\dfrac{i}{n}} => \dfrac{ 4 +3i }{2+0} => \dfrac{4}{2}+\dfrac{3i}{2} =>  $$

$$\mathnormal{L}\,=\, 2+\dfrac{3i}{2}$$

\begin{center}
\textbf{Series Complejas}
\end{center}

En los ejercicios \textbf{(\,7\,)} al \textbf{(\,11\,)} muestre si la \textcolor{mint}{{\it Serie Geométrica}} dada es Covergente ó Divergente. En caso de ser covergente, muestre a cuál número converge.\\

\textcolor{ao(english)}{(\,7\,)} $\bf{\displaystyle\sum\limits_{k\,=\,0}^{\infty}\,(1\,-\,i)^{k}}$.\\

\textcolor{ao(english)}{\ding{46}} Expandimos la serie geométrica para hallar el termino constante $a$.

$$=\,\textcolor{blue}{1}\,+\,\textcolor{blue}{1}\,(1\,-\,i)\,+\,\textcolor{blue}{1}\,(1\,-\,i)^{2}\,+\,...$$

$$a\,=\,1$$

\textcolor{ao(english)}{\ding{46}} Hallar el número al que converge si $|z|\,<\,1$.

$$|z|\,=\,|1\,-\,i|\,=\,\sqrt{1^{2}\,+\,(-\,1)^{2}}\,=\,\sqrt{2}\,\approx\,1.41$$

$\therefore\,|z|\,>\,1$ la Serie Geométrica es Divergente.\\

\textcolor{ao(english)}{(\,8\,)} $\bf{\displaystyle\sum\limits_{k\,=\,1}^{\infty}\,4\,i\,\left(\dfrac{1}{3}\right)^{k\,-\,1}}$.

$$k_{1}=4i $$

$$k_{2}=4i \times \dfrac{1}{3} = \dfrac{4i}{3}$$

$$k_{3}=4i \times \left(\dfrac{1}{3} \right)^{2} = \dfrac{4i}{9} $$

$$k_{4}=4i \times \left(\dfrac{1}{3} \right)^{3} = \dfrac{4i}{27} $$

$$ a=4i \quad z=\dfrac{1}{3} $$

\providecommand{\abs}[1]{\lvert#1\rvert} 

Comprobando si $\abs{z} < 1$
.\\

$$\sqrt{\left( \dfrac{1}{3} \right)^{2}}= \left( \dfrac{1}{3} \right) = 0.33$$

$$ Sn = \dfrac{a}{1-z} => \dfrac{4i}{1-\dfrac{1}{3}} => \dfrac{\dfrac{4i}{1}}{\dfrac{2}{3}} = \dfrac{12i}{2}$$

$$ Sn = \mathnormal{L}\,= 6i $$

\textcolor{ao(english)}{(\,9\,)} $\bf{\displaystyle\sum\limits_{k\,=\,1}^{\infty}\,\left(\dfrac{i}{2}\right)^{k}}$.\\

\textcolor{ao(english)}{\ding{46}} El termino constante $a$ de la serie geométrica es:

$$a\,=\,1$$\\

\textcolor{ao(english)}{\ding{46}} Hallar el número al que converge si $|z|\,<\,1$.

$$z\,=\,\dfrac{i}{2}$$

$$|z|\,=\,\left|\dfrac{i}{2}\right|\,=\,\dfrac{|i|}{|2|}\,=\,\dfrac{1}{2}\,=\,0,5$$

\textcolor{ao(english)}{\ding{46}} Como $|z|\,<\,1$, entonces la serie converge.

$$Sn\,=\,\dfrac{a}{1\,-\,z}\,=\,\dfrac{1}{1\,-\,\dfrac{i}{2}}\,=\,\dfrac{1}{\dfrac{2\,-\,i}{2}}\,=\,\dfrac{2}{2\,-\,i}\,\textcolor{red}{\times\,\dfrac{2\,+\,\,i}{2\,+\,i}}\,=\,\dfrac{4\,+\,2\,i}{5}$$

la serie converge a $L\,=\,\dfrac{4}{5}\,+\,\dfrac{2}{5}\,i$





\textcolor{ao(english)}{(\,10\,)} $\bf{\displaystyle\sum\limits_{k\,=\,0}^{\infty}\,\dfrac{1}{2}\,i^{k}}$.

\textcolor{ao(english)}{(\,11\,)} $\bf{\displaystyle\sum\limits_{k\,=\,0}^{\infty}\,3\,\left(\dfrac{2}{1\,+\,2\,i}\right)^{k}}$\\

\textcolor{ao(english)}{\ding{46}} El termino constante $a$ de la serie geométrica es:

$$a\,=\,3$$

\textcolor{ao(english)}{\ding{46}} Hallar el número al que converge si $|z|\,<\,1$.

$$z\,=\,\dfrac{2}{1\,+\,2\,i}\,\textcolor{red}{\times\,\dfrac{1\,-\,2\,i}{1\,-\,2\,i}}\,=\,\dfrac{2\,-\,4\,i}{5}$$

$$|z|\,=\,\left|\dfrac{2\,-\,4i}{5}\right|\,\dfrac{|2\,-\,4\,i|}{|5|}\,=\,\dfrac{\sqrt{2^{2}\,+\,(-\,4)^{2}}}{5}\,=\,\dfrac{\sqrt{4\,+\,16}}{5}\,=\,\dfrac{\sqrt{20}}{5}\,=\,\dfrac{2\,\sqrt{5}}{5}\,\approx\,0.9$$

\textcolor{ao(english)}{\ding{46}} Como $|z|\,<\,1$, entonces la serie converge.

$$Sn\,=\,\dfrac{a}{1\,-\,z}\,=\,\dfrac{3}{1\,-\,\left(\dfrac{2\,-\,4\,i}{5}\right)}\,=\,\dfrac{3}{\dfrac{5\,-\,2\,+\,4\,i}{5}}\,=\,\dfrac{15}{3\,+\,4\,i}\,\textcolor{red}{\times\,\dfrac{3\,-\,\,4\,i}{3\,-\,4\,i}}\,=\,\dfrac{45\,-\,60\,i}{25}\,=\,\dfrac{9\,-\,12\,i}{5}$$

la serie converge a $L\,=\,\dfrac{9}{5}\,-\,\dfrac{12}{5}\,i$

\end{document}