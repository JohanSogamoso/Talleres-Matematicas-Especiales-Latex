\documentclass[a4paper,11pt,openany]{book}
\usepackage[utf8]{inputenc}
\usepackage[left=2.54cm,top=2.54cm,right=2.54cm,bottom=2.54cm]{geometry}
\usepackage[spanish]{babel}
\usepackage{amsmath}
\usepackage{amsfonts}
\usepackage{amssymb}
\usepackage{graphicx}
\usepackage{color}
\usepackage[usenames,dvipsnames]{xcolor}
\usepackage{pifont}
\usepackage{marvosym}
\newtheorem{teo}{Teorema}
\newtheorem{ejemplo}{Ejemplo}
\newtheorem{defi}{Definición}
\newtheorem{coro}{Corolario}
\newtheorem{prueba}{Prueba}
\newtheorem{exmp}{Example}[section]
\newtheorem{ejer}{Ejercicio}[section]
\def\proof{\paragraph{\textsf{Demostración.} }}
\def\endproof{\hfill $\blacksquare$ \\}
\usepackage{multirow, array} % para las tablas
\usepackage{multirow}
\usepackage{tabularx}
\usepackage{float} % para usar [H]
\usepackage{tikz}
\usepackage[all]{xy}
\usepackage{cancel}
\usetikzlibrary{positioning}
\usepackage{enumitem}
\newcommand*{\itembolasazules}[1]{% bolas 3D
\footnotesize\protect\tikz[baseline=-3pt]%
\protect\node[scale=.7, circle, shade, ball
color=green]{\color{white}\Large\bf#1};}
\usepackage{tcolorbox} 
\tcbuselibrary{listingsutf8}
\newtcolorbox[auto counter,number within=section]{example}[2][]
{colback=green!5!white,colframe=green!75!black,fonttitle=\bfseries, title=Ejercicio~\thetcbcounter: #2,#1}
\usepackage{background}
\backgroundsetup{
placement=center,
angle=0,
scale=1.1,
contents= {{\includegraphics{HojaCuadriculada.png}}}
}
 
 
\begin{document}
\begin{titlepage}
 
\begin{center}
\vspace*{-1in}
\begin{figure}[htb]
\begin{center}
\includegraphics[width=7cm]{ETITC.png}
\end{center}
\end{figure}
 
 
{\sc \huge Escuela Tecnológica Instituto Técnico Central (ETITC)}\\
\vspace*{0.15in}
Facultad de sistemas\\
\vspace*{0.6in}
\begin{Large}
\textbf{Taller 4 : Funciones, Límites y Derivadas en los Complejos} \\
\textbf{Matem{\'a}ticas Especiales}\\
\end{Large}
\vspace*{0.3in}
\begin{large}
{\bf Autores} \\
 
\ 
 
Sergio Alejandro Enrrique Caballero Leon\\ 
Johan Alejandro Sogamoso Camacho \\
David Andrés Valero Vanegas \\
\end{large}
\vspace*{0.3in}
 
\end{center}
 
\begin{center}
{\bf Presentado a:} \\
 
\ 
 
Carlos Romero \\
 
\
 
Bogot{\'a}, Octubre de 2022.
\end{center}
 
\end{titlepage}

\newpage

\definecolor{ao(english)}{rgb}{0.0, 0.5, 0.0}

\graphicspath{ {images/} }

\begin{center}
\textbf{Funciones}
\end{center}

En los ejercicio \textbf{(\,1\,)} al \textbf{(\,5\,)} sea $f(z)$ la función que que actúa sobre el conjunto dado $\mathnormal{S}$. Hallar la imágen $\mathnormal{S\,'}$ correspondiente a cada conjunto y graficar su respectivo mapeo.\\

\textcolor{ao(english)}{(\,1\,)} $\mathnormal{S}$ es $\boxed{\bf{y\,=\,1\,-\,x}}$ y $\boxed{\bf{f(z)\,=\,z\,+\,i\,(i\,-\,2)}}$.

\textcolor{ao(english)}{(\,2\,)} $\mathnormal{S}$ es $\boxed{\bf{1\,<\,\mathnormal{Re}(z)\,<\,4}}$ y $\boxed{\bf{f(z)\,=\,3\,z}}$.

\textcolor{ao(english)}{(\,3\,)} $\mathnormal{S}$ es $\boxed{\bf{y\,=\,-\,2}}$ y $\boxed{\bf{f(z)\,=\,2\,i\,+\,z\,(1\,+\,i)}}$.

\textcolor{ao(english)}{(\,4\,)} $\mathnormal{S}$ es $\boxed{\bf{-\,1\,<\,\mathnormal{Im}(z)\,<\,2}}$ y $\boxed{\bf{f(z)\,=\,i\,z\,+\,4}}$.

\textcolor{ao(english)}{(\,5\,)} $\mathnormal{S}$ es $\boxed{\bf{\mathnormal{Re}(z)\,>\,3}}$ y $\boxed{\bf{f(z)\,=\,z^{2}}}$.

\begin{center}
\textbf{Límites}
\end{center}

En los ejercicio \textbf{(\,6\,)} al \textbf{(\,11\,)} use las \textbf{propiedades de los límites} para calcular los límites indicados.\\

\textcolor{ao(english)}{(\,6\,)} $\bf{\displaystyle\lim_{z \to 2\,i}\,(z^{2}\,-\,\overline{z})}$

\textcolor{ao(english)}{(\,7\,)} $\bf{\displaystyle\lim_{z \to 3\,i}\,\dfrac{\mathnormal{Im}(z^{2})}{z\,+\,\mathnormal{Re}(z)}}$

\textcolor{ao(english)}{(\,8\,)} $\bf{\displaystyle\lim_{z \to 1\,+\,i}\,\dfrac{z^{2}\,+\,1}{z^{2}\,-\,1}}$

\textcolor{ao(english)}{(\,9\,)} $\bf{\displaystyle\lim_{z \to 3\,+\,i\,\sqrt{2}}\,\dfrac{z\,+\,3\,-\,i\,\sqrt{2}}{z^{2}\,+\,6\,z\,+\,11}}$

\textcolor{ao(english)}{(\,10\,)} $\bf{\displaystyle\lim_{z \to 2}\,\dfrac{z\,2\,-\,5\,z\,+\,6}{z^{2}\,-\,4}}$

\textcolor{ao(english)}{(\,11\,)} $\bf{\displaystyle\lim_{z \to 3\,i}\,\dfrac{z^{4}\,+\,10\,z^{2}\,+\,9}{z^{2}\,-\,4\,i\,z\,-\,3}}$

\begin{center}
\textbf{Derivadas}
\end{center}

\textcolor{ao(english)}{(\,12\,)} Use la fórmula de la \textbf{Definición de Derivada} mostrada a continuación:\\

$$\bf{f'\,(z_{0})\,=\,\displaystyle\lim_{\Delta\,z \to 0}\,\dfrac{f(z_{0}\,+\,\Delta\,z)\,-\,f(z_{0})}{\Delta\,z}}$$\\

para calcular la derivada de la función $f(z)\,=\,z\,\mathnormal{Im}(z)$ en los puntos $z_{0}\,=\,0$, $z_{0}\,=\,1\,+\,i$ y $z_{0}\,=\,-\,2\,-\,i$. ¿Ésta función es Analítica?\\

En los ejercicio \textbf{(\,13\,)} al \textbf{(\,15\,)} derive las siguientes funciones usando las \textbf{Reglas de la diferencición} (simplifique tanto como sea posible).\\

\textcolor{ao(english)}{(\,13\,)} $\bf{f(z)\,=\,-\,5\,z^{2}\,+\,\dfrac{2\,+\,i}{z^{2}}}$

\textcolor{ao(english)}{(\,14\,)} $\bf{f(z)\,=\,(i\,z^{2}\,+\,3\,z)^{5}}$

\textcolor{ao(english)}{(\,15\,)} $\bf{f(z)\,=\,(z^{2}\,+\,2\,z\,-\,7\,i)^{2}\,(z^{4}\,-\,4\,i\,z)^{3}}$

\begin{center}
\textbf{Ecuaciones de Cauchy-Riemann}
\end{center}

En los ejercicio \textbf{(\,16\,)} al \textbf{(\,20\,)}, use las Ecuaciones de Cauchy-Riemann para mostrar si las siguientes funciones complejas son o no Analíticas.

\textcolor{ao(english)}{(\,16\,)} $\bf{f(z)\,=\,z^{2}\,+\,5\,i\,z\,+\,3\,-\,i}$

\textcolor{ao(english)}{(\,17\,)} $\bf{f(z)\,=\,e^{2\,x}\,(\cos\,y\,+\,i\,\sin\,y)}$

\textcolor{ao(english)}{(\,18\,)} $\bf{f(z)\,=\,3\,z^{2}\,+\,5\,z\,-\,6\,i}$

\textcolor{ao(english)}{(\,19\,)} $\bf{f(z)\,=\,4\,z\,-\,6\,\overline{z}\,+\,3}$

\textcolor{ao(english)}{(\,20\,)} $\bf{f(z)\,=\,e^{-\,x}\,(\cos\,y\,-\,i\,\sin\,y)}$

\end{document}