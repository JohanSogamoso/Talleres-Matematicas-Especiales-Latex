\documentclass[a4paper,11pt,openany]{book}
\usepackage[utf8]{inputenc}
\usepackage[left=2.54cm,top=2.54cm,right=2.54cm,bottom=2.54cm]{geometry}
\usepackage[spanish]{babel}
\usepackage{amsmath}
\usepackage{amsfonts}
\usepackage{amssymb}
\usepackage{graphicx}
\usepackage{color}
\usepackage[usenames,dvipsnames]{xcolor}
\usepackage{pifont}
\usepackage{marvosym}
\newtheorem{teo}{Teorema}
\newtheorem{ejemplo}{Ejemplo}
\newtheorem{defi}{Definición}
\newtheorem{coro}{Corolario}
\newtheorem{prueba}{Prueba}
\newtheorem{exmp}{Example}[section]
\newtheorem{ejer}{Ejercicio}[section]
\def\proof{\paragraph{\textsf{Demostración.} }}
\def\endproof{\hfill $\blacksquare$ \\}
\usepackage{multirow, array} % para las tablas
\usepackage{multirow}
\usepackage{tabularx}
\usepackage{float} % para usar [H]
\usepackage{tikz}
\usepackage[all]{xy}
\usepackage{cancel}
\usetikzlibrary{positioning}
\usepackage{enumitem}
\newcommand*{\itembolasazules}[1]{% bolas 3D
\footnotesize\protect\tikz[baseline=-3pt]%
\protect\node[scale=.7, circle, shade, ball
color=green]{\color{white}\Large\bf#1};}
\usepackage{tcolorbox} 
\tcbuselibrary{listingsutf8}
\newtcolorbox[auto counter,number within=section]{example}[2][]
{colback=green!5!white,colframe=green!75!black,fonttitle=\bfseries, title=Ejercicio~\thetcbcounter: #2,#1}
\usepackage{background}
\backgroundsetup{
placement=center,
angle=0,
scale=1.1,
contents= {{\includegraphics{HojaCuadriculada.png}}}
}
 
 
\begin{document}
\begin{titlepage}
 
\begin{center}
\vspace*{-1in}
\begin{figure}[htb]
\begin{center}
\includegraphics[width=7cm]{ETITC.png}
\end{center}
\end{figure}
 
 
{\sc \huge Escuela Tecnológica Instituto Técnico Central (ETITC)}\\
\vspace*{0.15in}
Facultad de sistemas\\
\vspace*{0.6in}
\begin{Large}
\textbf{Taller 2: Forma Polar, Potencias y Raíces} \\
\textbf{Matem{\'a}ticas Especiales}\\
\end{Large}
\vspace*{0.3in}
\begin{large}
{\bf Autores} \\
 
\ 
 
Sergio Alejandro Enrrique Caballero Leon\\ 
Johan Alejandro Sogamoso Camacho \\
David Andrés Valero Vanegas \\
\end{large}
\vspace*{0.3in}
 
\end{center}
 
\begin{center}
{\bf Presentado a:} \\
 
\ 
 
Carlos Romero \\
 
\
 
Bogot{\'a}, Septiembre de 2022.
\end{center}
 
\end{titlepage}

\newpage

\definecolor{ao(english)}{rgb}{0.0, 0.5, 0.0}

\begin{center}
\textbf{Multiplicación}
\end{center}

En cada uno de los ejercicios \textbf{(\,1\,)} al \textbf{(\,3\,)} realizar lo siguiente:\\

\textbf{a\,)} Calcular el producto de los números complejos $\bf{z_{1}\,\cdot\,z_{2}}$, usando la fórmula de la multiplicación en forma polar.\\

\textbf{b\,)} Escriba \textbf{el resultado del producto} en la forma cartesiana $\bf{z\,=\,x\,+\,i\,y}$ y en la forma exponencial $\bf{z\,=\,r\,e^{i\,\theta}}$ ó $\bf{z\,=\,r\,exp\,(i\,\theta)}$.\\

\textbf{c\,)} Realizar en un plano complejo la gráfica de $\bf{z_{1}}$, $\bf{z_{2}}$ y $\bf{z_{1}\,\cdot\,z_{2}}$.\\

\textcolor{ao(english)}{(\,1\,)} $\bf{z_{1}\,=\,4\,-\,4\,i \quad;\quad z_{2}\,=\,6\,-\,i\,6\,\sqrt{3}}$

\textcolor{ao(english)}{(\,2\,)} $\bf{z_{1}\,=\,2\,-\,i \quad;\quad z_{2}\,=\,-\,2\,+\,2\,i}$

\textcolor{ao(english)}{(\,3\,)} $\bf{z_{1}\,=\,1\,-\,i \quad;\quad z_{2}\,=\,1\,-\,i\,\sqrt{3}}$

\begin{center}
\textbf{División}
\end{center}

En cada uno de los ejercicios \textbf{(\,4\,)} al \textbf{(\,6\,)} realizar lo siguiente:\\

\textbf{a\,)} Calcule $\bf{z_{1}\,\div\,z_{2}}$, usando la fórmula de la división en forma polar.\\

\textbf{b\,)} Escriba \textbf{el resultado de la división} en la forma cartesiana $\bf{z\,=\,x\,+\,i\,y}$ y en la forma exponencial $\bf{z\,=\,r\,e^{i\,\theta}}$ ó $\bf{z\,=\,r\,exp\,(i\,\theta)}$.\\

\textbf{c\,)} Realizar en un plano complejo la gráfica de $\bf{z_{1}}$, $\bf{z_{2}}$ y $\bf{z_{1}\,\div\,z_{2}}$.\\

\textcolor{ao(english)}{(\,4\,)} $\bf{z_{1}\,=\,2\,-\,2\,i \quad;\quad z_{2}\,=\,3\,-\,i\,3\,\sqrt{3}}$

\textcolor{ao(english)}{(\,5\,)} $\bf{z_{1}\,=\,-\,1\,-\,i\,\sqrt{3} \quad;\quad z_{2}\,=\,-\,4\,+\,4\,i}$

\textcolor{ao(english)}{(\,6\,)} $\bf{z_{1}\,=\,2\,-\,2\,i \quad;\quad z_{2}\,=\,2\,-\,i\,2\,\sqrt{3}}$

\begin{center}
\textbf{Potencias Enteras de Números Complejos}
\end{center}

En cada uno de los ejercicios \textbf{(\,7\,)} al \textbf{(\,8\,)} realizar lo siguiente:\\

\textbf{a\,)} Calcule las potencias indicadas en forma exponencial.\\

\textbf{b\,)} Escriba el resultado final en la forma cartesiana $\bf{z^{n}\,=\,x\,+\,i\,y}$.\\

\textbf{c\,)} Dibujo en uno dos planos complejos los números, $\bf{z}$ y $\bf{z^{n}}$.\\

\textcolor{ao(english)}{(\,7\,)} $\bf{\left(1\,+\,i\,\sqrt{3}\right)^{9}}$

\textcolor{ao(english)}{(\,8\,)} $\bf{\left(2\,-\,2\,i\right)^{5}}$

\begin{center}
\textbf{Raíces de Números Complejos}
\end{center}

Hallar las raíces indicadas a continuación y realizar su correspondiente gráfica en el plano complejo.\\

\textcolor{ao(english)}{(\,9\,)} $\bf{\sqrt[3]{8}}$

\textcolor{ao(english)}{(\,10\,)} $\bf{\sqrt[6]{-\,27\,i}}$

\textcolor{ao(english)}{(\,11\,)} $\bf{\sqrt{4\,\sqrt{2}\,+\,i\,4\,\sqrt{2}}}$

\textcolor{ao(english)}{(\,12\,)} $\left(-\,16\,+\,i\,16\,\sqrt{3}\right)^{1\,/\,5}$

\end{document}